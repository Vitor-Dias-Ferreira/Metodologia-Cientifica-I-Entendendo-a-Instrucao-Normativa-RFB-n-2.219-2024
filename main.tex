\documentclass{beamer}
\usepackage[utf8]{inputenc}
\usepackage[brazil]{babel}
\usepackage[alf]{abntex2cite}
\usepackage{hyperref}
\usepackage{graphicx}
\usepackage{booktabs} 
\usepackage{ragged2e} 
\usepackage{type1cm}
\usepackage{eso-pic}
\usepackage{color}
\usepackage{tikz} 
\usepackage{eso-pic} 
\usetikzlibrary{calc}
\usepackage{pgfplots}
\usepgfplotslibrary{dateplot}
\usetikzlibrary{positioning, shadows}
\usepackage{ragged2e}
\usepackage[T1]{fontenc}
\usepackage{adjustbox}
\usetikzlibrary{shapes, arrows, positioning}

\definecolor{rfblue}{RGB}{0,74,153}
\definecolor{rfgreen}{RGB}{0,109,44}
\definecolor{rfred}{RGB}{200,16,46}
\definecolor{lightgray}{RGB}{240,240,240}

\setbeamercolor{palette primary}{bg=rfblue,fg=white}
\setbeamercolor{palette secondary}{bg=rfblue,fg=white}
\setbeamercolor{palette tertiary}{bg=rfblue,fg=white}
\setbeamercolor{progress bar}{fg=rfred,bg=lightgray}
\setbeamercolor{title separator}{fg=rfred}
\setbeamercolor{block title}{bg=rfblue,fg=white}
\setbeamercolor{block body}{bg=lightgray,fg=black}


% Watermark command - ADD THIS TO YOUR PREAMBLE
\newcommand{\watermark}{
\begin{tikzpicture}[remember picture, overlay]
\node[opacity=0.05] at ($(current page.east)$) 
    {\includegraphics[width=8cm]{icmc_logo.png}}; % Adjust opacity/size here
\end{tikzpicture}
}

\AddToShipoutPictureFG{\watermark}

\definecolor{azulrf}{RGB}{0,74,153}
\definecolor{verderf}{RGB}{0,109,44}
\setbeamercolor{structure}{fg=azulrf}

\usetheme{metropolis}
\title{Entendendo a \\ Instrução Normativa RFB nº 2.219/2024}
\author{
    Alexander Kahleul \\
    Fernando Ramos \\
    Gabriel Maia \\
    Vitor Dias
}
\institute{Universidade de São Paulo - USP \\
Instituto de Ciências Matemáticas e de Computação - ICMC
}

\date{\today}

\begin{document}

\begin{frame}
  \titlepage
\end{frame}

\begin{frame}
    \section{Introdução}
\end{frame}

\begin{frame}{Introdução}
      \begin{columns}[T]
        \column{0.5\textwidth}
        \centering
        \textbf{Banco Central (BCB)}
        \begin{itemize}
            \item Regulamentação do sistema financeiro
            \item Condução da política monetária
            \item Controle da inflação
            \item Operacionalização do Pix
        \end{itemize}
        
        \column{0.5\textwidth}
        \centering
        \textbf{Receita Federal (RFB)}
        \begin{itemize}
            \item Fiscalização tributária
            \item Combate à sonegação fiscal
            \item Monitoramento via e-Financeira
            \item Coleta de dados financeiros
        \end{itemize}
    \end{columns}
    
    \vspace{0.7cm}
        \begin{block}{}
            \begin{tikzpicture}[
            node distance=4.5cm,
            box/.style={draw, rounded corners, align=center, minimum width=3.2cm, minimum height=0.8cm, fill=white, font=\footnotesize},
            arrow/.style={<->, >=stealth, thick, shorten >=2pt, shorten <=47pt}
            ]
                \node[box, fill=azulrf!20] (bcb) {Banco Central};
                \node[box, fill=azulrf!30, right=of bcb] (rfb) {Receita Federal}; 
                \draw[arrow] (bcb) ++ -- ++ (rfb)
                node[above, midway, font=\footnotesize, xshift=0.6cm] {e-Financeira}
                
            \end{tikzpicture}
        \end{block}
\end{frame}

\begin{frame}
    \frametitle{Introdução}
    \begin{block}{e-Financeira}
    \begin{itemize}
        \item Plataforma eletrônica para recebimento de informações financeiras.
        \item Monitoramento de operações suspeitas (fraudes, sonegação)
        \item Fontes: Bancos, fintechs, instituições de pagamento, etc
    \end{itemize}
    \end{block}
\end{frame}

\begin{frame}{Introdução}
    \begin{block}{\textit{Fintechs}}
    \begin{itemize}
        \item Empresas do setor financeiro que se utilizam de tecnologia para otimizar serviços e reduzir custos operacionais.
        \item Regulamento específico.
    \end{itemize}
    \end{block}

    \begin{block}{\textit{O Problema}}
    \begin{itemize}
        \item Como fiscalizar?
    \end{itemize}
    \end{block}
\end{frame}

\begin{frame}{Abordagem Metodológica}
\fontsize{7}{11}
    \begin{columns}[T]
    \column{0.5\textwidth}
    \begin{block}{Desenho da Pesquisa}
    \begin{itemize}
        \item \textbf{Natureza:} Estudo de caso
        \item \textbf{Objeto:} Ciclo de vida da IN 2.219/2024
        \item \textbf{Abordagem:} Qualitativa
        \item \textbf{Período:} 2020-2025
    \end{itemize}
    \end{block}
    
    \begin{block}{Fontes Primárias}
    \begin{itemize}
        \item Texto da IN 2.219/2024
        \item Comunicações oficiais (RFB, BCB)
        \item Dados estatísticos do Pix
        \item Relatórios regulatórios
    \end{itemize}
    \end{block}
    
    \column{0.5\textwidth}
    \begin{block}{Fontes Secundárias}
    \begin{itemize}
        \item Reportagens jornalísticas
        \item Artigos especializados
        \item Materiais audiovisuais
        \item Análises de impacto
    \end{itemize}
    \end{block}
    
    \begin{block}{Técnicas Analíticas}
    \begin{itemize}
        \item Análise documental
        \item Análise de conteúdo
        \item Análise estatística
        \item Análise de discurso
    \end{itemize}
    \end{block}
    \end{columns}
\end{frame}

\begin{frame}{Processo Analítico Compacto}
    \centering
    \scalebox{0.60}{ % Reduz o tamanho do diagrama
    \begin{tikzpicture}[
        node distance=0.5cm,
        box/.style={draw, rounded corners, align=center, minimum width=3.2cm, minimum height=0.8cm, fill=white, font=\footnotesize},
        arrow/.style={->, >=stealth, thick, shorten >=2pt, shorten <=2pt}
    ]
    
    % Etapas principais em linha reta
    \node[box, fill=azulrf!20] (coleta) {Coleta de dados};
    \node[box, fill=azulrf!30, right=of coleta] (organiz) {Organização};
    \node[box, fill=verde!20, right=of organiz] (analise) {Análise};
    \node[box, fill=verde!30, right=of analise] (triang) {Triangulação};
    \node[box, fill=vermelho!20, right=of triang] (valid) {Validação};
    
    % Conexões principais
    \draw[arrow] (coleta) -- (organiz);
    \draw[arrow] (organiz) -- (analise);
    \draw[arrow] (analise) -- (triang);
    \draw[arrow] (triang) -- (valid);
    \draw[arrow] (valid) -- ++(0,-1.5) -| (coleta); % Ciclo de retorno
    
    % Detalhes abaixo das etapas
    \node[below=0.1cm of coleta, text width=3.2cm, align=center, font=\tiny] 
        {documentos e\\ reportagens};
    \node[below=0.1cm of organiz, text width=3.2cm, align=center, font=\tiny] 
        {Categorização temática \\ Codificação};
    \node[below=0.1cm of analise, text width=3.2cm, align=center, font=\tiny] 
        {Qualitativa + Quantitativa \\ Argumentos + Dados};
    \node[below=0.1cm of triang, text width=3.2cm, align=center, font=\tiny] 
        {Integração de métodos \\ Validação cruzada};
    \node[below=0.1cm of valid, text width=3.2cm, align=center, font=\tiny] 
        {Conclusões \\ Recomendações};
    
    % Legenda explicativa
    \node[below=1.2cm of analise, text width=10cm, align=center, font=\small] (leg1) 
        {\textcolor{azulrf}{Análise qualitativa: Argumentos, narrativas e cronologia}};
    \node[below=0.1cm of leg1, text width=10cm, align=center, font=\small] (leg2) 
        {\textcolor{verde}{Análise quantitativa: Dados do Pix e estatísticas regulatórias}};
    \end{tikzpicture}
    }
    
    % Explicação adicional
    \begin{block}{Fluxo do Processo}
        \begin{itemize}
        \item \textcolor{azulrf}{\textbf{Linear:}} Sequência lógica de 5 etapas fundamentais
        \item \textcolor{verde}{\textbf{Cíclico:}} Validação retroalimenta o processo de coleta
        \item \textbf{Integrado:} Combinação permanente de abordagens qualitativa e quantitativa
        \end{itemize}
        \end{block}
    \end{frame}
    
    % Diagrama de Triangulação corrigido
    \begin{frame}{Triangulação de Dados}
    \centering
    \begin{tikzpicture}[
        rot/.style={ellipse, draw, thick, minimum width=3cm, minimum height=1.8cm, align=center, font=\small}
    ]
    
    % Posicionamento otimizado
    \node[rot, fill=azulrf!20] (normativa) at (0,0) {Análise\\Normativa};
    \node[rot, fill=verde!20] (dados) at (4,0) {Dados\\Quantitativos};
    \node[rot, fill=vermelho!20] (discurso) at (0,-3) {Análise de\\Discurso};
    \node[rot, fill=azulrf!30] (contexto) at (4,-3) {Contexto\\Regulatório};
    
    % Conexões otimizadas
    \draw[->, thick] (normativa) -- node[midway, above, font=\scriptsize] {1} (dados);
    \draw[->, thick] (dados) -- node[midway, right, font=\scriptsize] {2} (discurso);
    \draw[->, thick] (discurso) -- node[midway, below, font=\scriptsize] {3} (contexto);
    \draw[->, thick] (contexto) -- node[midway, left, font=\scriptsize] {4} (normativa);
    \draw[->, thick] (normativa) to[out=240, in=120] node[midway, left, font=\scriptsize] {5} (discurso);
    \draw[->, thick] (dados) to[out=300, in=60] node[midway, right, font=\scriptsize] {6} (contexto);
    
    % Legenda
    \node[below=0.5cm of contexto, text width=8cm, align=center, font=\small] 
        {\textbf{Integração metodológica para compreensão multidimensional do fenômeno}};
    \end{tikzpicture}
\end{frame}

\begin{frame}
    \section{Análise da IN RFB 2.219/2024}
\end{frame}

\begin{frame}{Instruções Normativas: Conceito e Características}
\fontsize{8}{9}
\scalebox{0.85}{
\begin{columns}[T]
\column{0.6\textwidth}
\begin{block}{Definição e Natureza Jurídica}
\begin{itemize}
    \item \textbf{Ato administrativo} de caráter normativo
    \item \textbf{Hierarquicamente inferior} a:
    \begin{itemize}
        \item Constituição Federal
        \item Leis (complementares/ordinárias)
        \item Decretos
    \end{itemize}
    \item \textbf{Não pode} criar, modificar ou extinguir direitos
    \item Função: \textbf{regulamentar e detalhar} aplicação de leis
\end{itemize}
\end{block}

\begin{block}{Função Primordial}
\begin{itemize}
    \item \textbf{Traduzir comandos abstratos} da lei
    \item Estabelecer \textbf{rotinas e procedimentos operacionais}
    \item Garantir \textbf{fiel execução} da legislação
    \item Expedidas por \textbf{autoridades administrativas} (ex: Secretário da RFB)
\end{itemize}
\end{block}

\column{0.4\textwidth}
\centering
\begin{tikzpicture}[
    nivel/.style={draw, rounded corners, align=center, minimum width=3.5cm, minimum height=0.8cm, font=\footnotesize}
]

% Hierarquia normativa
\node[nivel, fill=verde!30] (const) {1. Constituição Federal};
\node[nivel, fill=azulrf!20, below=0.2cm of const] (leis) {2. Leis Complementares/Ordinárias};
\node[nivel, fill=azulrf!30, below=0.2cm of leis] (decretos) {3. Decretos};
\node[nivel, fill=vermelho!20, below=0.2cm of decretos] (ins) {4. Instruções Normativas};

% Setas hierárquicas
\draw[->, thick] (const) -- (leis);
\draw[->, thick] (leis) -- (decretos);
\draw[->, thick] (decretos) -- (ins);
\end{tikzpicture}

\vspace{0.5cm}
\begin{alertblock}{Controvérsias Jurídicas}
\fontsize{8}{7}
\begin{itemize}
    \item Risco de \textbf{extrapolar limites} da lei
    \item Possibilidade de \textbf{contestação judicial}
    \item Tensão entre \textbf{regulamentação × inovação}
    \item Fonte de \textbf{insegurança jurídica}
\end{itemize}
\end{alertblock}
\end{columns}}
\end{frame}


\begin{frame}
    \frametitle{O que a IN 2.219 propunha?}
    \begin{table}
    \centering
    \begin{tabular}{lcc}
    \toprule
    \textbf{Critério} & \textbf{Limite Anterior} & \textbf{Limite Novo} \\
    \midrule
    Pessoas Físicas (PF) & R\$ 2.000/mês & {\textbf{R\$ 5.000/mês}} \\
    Pessoas Jurídicas (PJ) & R\$ 6.000/mês & {\textbf{R\$ 15.000/mês}} \\
    \bottomrule
    \end{tabular}
    \end{table}
    \vspace{0.5cm}
    
    \begin{alertblock}{Ampliação do Rol de Obrigados:}
    \begin{itemize}
        \item Instituições financeiras não tradicionais são obrigadas a prestar informações
    \end{itemize}
    \end{alertblock}
\end{frame}

%---- OBJETIVOS DA NORMA
\begin{frame}
    \frametitle{Objetivos da Norma}
    \begin{columns}[T]
        \column{0.5\textwidth}
        \centering
        \includegraphics[width=0.67\linewidth]{pix-logo.png} \\
        \textbf{Adequação ao cenário digital}
        \begin{itemize}
            \item Volume maior de transações
            \item Crescimento de fintechs
            \item Popularização do Pix
        \end{itemize}
        
        \column{0.5\textwidth}
        \centering
        \includegraphics[width=0.4\linewidth]{bcb-logo.png} \\
        \textbf{Reforço do controle fiscal}
        \begin{itemize}
            \item Identificação de operações atípicas
            \item Combate à sonegação
            \item Otimização de cruzamento de dados
        \end{itemize}
    \end{columns}
\end{frame}
% --- POLEMICAS
\begin{frame}
    \frametitle{Impactos}
    \begin{columns}[T]
        \column{0.5\textwidth}
        \begin{block}{Vantagens}
        \begin{itemize}
            \item \checkmark~Maior eficiência fiscal
            \item \checkmark~Atualização de parâmetros
            \item \checkmark~Combate a ilícitos financeiros
        \end{itemize}
        \end{block}
        
        \column{0.5\textwidth}
        \begin{alertblock}{Preocupações}
        \begin{itemize}
            \item \textbf{Privacidade} de dados
            \item \textbf{Sigilo bancário}
            \item Comunicação pública
            \item Desinformação
        \end{itemize}
        \end{alertblock}
\end{columns}

\vspace{0.6cm}
\begin{exampleblock}{Posição oficial da RFB}
    \vspace{0.2cm}
    "A IN não cria tributos, apenas atualiza mecanismos de fiscalização existentes"
\end{exampleblock}
\end{frame}

%-----------------------------------------
\begin{frame}{Impactos e Desinformação: O Caso da IN RFB 2.219/2024}
\centering
\fontsize{8}{11}
% Título da seção

% Conteúdo principal em duas colunas
\begin{columns}[T]

% Coluna 1: O processo de desinformação
\column{0.5\textwidth}
\begin{block}{\textcolor{red}{A Propagação da Desinformação}}
\begin{itemize}
\item \textbf{Fase técnica (2024):} \\
  Preparação silenciosa por especialistas
\item \textbf{Virada (Jan/2025):} \\
  Norma rebatizada como "imposto sobre o Pix"
\item \textbf{Vídeo viral (14/Jan):} \\
  Deputado Nikolas Ferreira - 100M+ visualizações
\item \textbf{Narrativa simplista:} \\
  "Governo quer taxar suas transferências"
\end{itemize}
\end{block}



% Coluna 2: Impactos e consequências
\column{0.5\textwidth}
\begin{alertblock}{\textcolor{red}{Consequências Imediatas}}
\begin{itemize}
\item \textbf{Revogação da norma:} \\
  IN RFB nº 2.247/2025 em 15/Jan
\item \textbf{Precedente perigoso:} \\
  Políticas técnicas derrotadas por desinformação
\item \textbf{Perda oportunidade:} \\
  Modernização fiscal adiada
\item \textbf{Descrédito institucional:} \\
  Autoridades técnicas desautorizadas
\end{itemize}
\end{alertblock}

\end{columns}

% Declaração de impacto


\end{frame}
%----------------------------------------
\begin{frame}{Impactos e Desinformação: O Caso da IN RFB 2.219/2024}

\begin{exampleblock}{Mecanismo da Crise}
\begin{itemize}
\item Informação técnica complexa \\
  $\Downarrow$ \\
  Simplificação distorcida \\
  $\Downarrow$ \\
  Viralização em redes sociais \\
  $\Downarrow$ \\
  Pressão política massiva
\end{itemize}
\end{exampleblock}

\begin{exampleblock}{}
\fontsize{8}{8}
\centering
\textbf{O vídeo funcionou como catalisador que tornou o custo político de manter a IN proibitivamente alto} \\

\end{exampleblock}

\end{frame}
%----------------------------------------
\begin{frame}{Cronologia dos principais eventos}
        \begin{table}[ht]
        
      \centering
      \caption{Cronologia dos principais eventos relacionados à IN RFB nº 2.219/2024}
      \label{tab:cronologia-in-rfb}
      \scalebox{0.7}{
      \begin{tabular}{|p{3cm}|p{10cm}|}
        \hline
        \textbf{Data} & \textbf{Evento} \\ \hline
        18/09/2024 & Publicação da IN RFB nº 2.219/2024 no Diário Oficial da União. \\ \hline
        01/01/2025 & Entrada em vigor das novas regras da IN, sem grande repercussão inicial. \\ \hline
        14/01/2025 & Deputado Nikolas Ferreira (PL) publica um vídeo crítico à medida, que viraliza massivamente, alcançando centenas de milhões de visualizações. \\ \hline
        15/01/2025 & Pressionado pela repercussão negativa, o Governo Federal anuncia a revogação da norma. A IN RFB nº 2.247/2025 é publicada em edição extra do Diário Oficial da União. \\ \hline
        16/01/2025 & Como medida adicional de contenção de danos, o Governo edita a Medida Provisória nº 1.288/2025 para reafirmar a gratuidade e a não tributação do Pix. \\ \hline
      \end{tabular}}
    \end{table}
\end{frame}


\begin{frame}{Crise, Resposta e Revogação da IN RFB 2.219/2024}
    \centering
    
    
    \box
    \scalebox{0.9}{
    % Linha do tempo simples
    \vspace{0.5cm}
    
    \begin{columns}[T]
    % Coluna 1: Desinformação
    \column{0.25\textwidth}
    \centering
    \textcolor{orange}{\textbf{Onda de Desinformação}}
    \begin{itemize}
        \scriptsize
        \item ``Novo imposto do Pix''
        \item ``Fim do sigilo bancário''
    \end{itemize}
    \vspace{0.2cm}
    \footnotesize 10/01/2025
    
    % Seta entre colunas
    \column{0.05\textwidth}
    \centering
    \Huge $\rightarrow$
    
    % Coluna 2: Resposta Oficial
    \column{0.25\textwidth}
    \centering
    \textcolor{orange}{\textbf{Resposta Oficial}}
    \vspace{0.5cm}
    \begin{itemize}
        \scriptsize
        \item Notas de esclarecimento
        \item Posts em canais
        \item \textbf{}{Não cria tributos}
        \item \textbf{Limites aumentados}
    \end{itemize}
    \vspace{0.2cm}
    \footnotesize 12/01/2025
    
    % Seta entre colunas
    \column{0.05\textwidth}
    \centering
    \Huge $\rightarrow$
    
    % Coluna 3: Ineficácia
    \column{0.25\textwidth}
    \centering
    \textcolor{orange}{\textbf{Ineficácia}}
    \vspace{1.0cm}
    \begin{itemize}
        \scriptsize
        \item Comunicação insuficiente
        \item Não surtiu efeito
    \end{itemize}
    \vspace{0.2cm}
    \footnotesize 14/01/2025
    
    % Seta entre colunas
    \column{0.05\textwidth}
    \centering
    \Huge $\rightarrow$
    
    % Coluna 4: Revogação
    \column{0.25\textwidth}
    \centering
    \textcolor{red}{\textbf{Revogação}}
    \vspace{1.0cm}
    \begin{itemize}
        \scriptsize
        \item 15/01/2025
        \item IN RFB nº 2.247/2025
    \end{itemize}
    \vspace{0.2cm}
    \footnotesize 15/01/2025
    \end{columns}
        }
    
\end{frame}

\begin{frame}{Crise, Resposta e Revogação da IN RFB 2.219/2024}
    \begin{figure}
        \centering
        \includegraphics[width=0.65\linewidth]{cnn_noticia.png}
        \caption{Notícia CNN sobre o tema}
        \label{fig:enter-label}
    \end{figure}
\end{frame}

\begin{frame}{Dados do Pix: Evolução Histórica (2020-2023)}
\fontsize{7}{8}
\centering

\includegraphics[width=0.7\textwidth]{transpix_2020.png}
\begin{block}{Relevância para o estudo}
\begin{itemize}
    \item Crescimento exponencial das transações digitais
    \item Diversificação de métodos de iniciação
    \item Contexto para atualização dos limites de reporte
\end{itemize}
\end{block}

\end{frame}

\begin{frame}{Dados do Pix: Projeção (2023-2025)}
\fontsize{7}{8}
\centering
\includegraphics[width=0.8\textwidth]{transaes-pix.png}

\begin{block}{Implicações regulatórias}
\begin{itemize}
    \item Volume projetado justifica atualização dos limites
    \item Necessidade de acompanhamento tecnológico
    \item Desafios para identificação de operações atípicas
\end{itemize}
\end{block}
\end{frame}



% ---- CONCLUSÃO ----
\begin{frame}
    \section{Conclusão}
\end{frame}

\begin{frame}{Revisão}
\begin{columns}[T]
\column{0.5\textwidth}
\begin{block}{Objetivos originais}
\begin{itemize}
    \item Atualizar limites de reporte financeiro
    \item PF: de R\$ 2.000 para R\$ 5.000
    \item PJ: de R\$ 6.000 para R\$ 15.000
    \item Modernizar fiscalização tributária
\end{itemize}
\end{block}

\column{0.5\textwidth}
\begin{block}{Reação e resultado}
\begin{itemize}
    \item Forte reação social e política
    \item Desinformação sobre criação de novos tributos
    \item Normativa revogada em poucos dias
    \item Retorno aos limites anteriores
\end{itemize}
\end{block}
\end{columns}
\end{frame}

\begin{frame}{Principais Conclusões}
\fontsize{9}{11}\selectfont
\vspace{0.3cm}
\begin{block}{1. Desafios da regulação digital}
\begin{itemize}
    \item Dificuldade em normatizar setores em rápida transformação
    \item Velocidade tecnológica \texttimes{} lentidão regulatória
    \item Necessidade de modelos mais adaptativos
\end{itemize}
\end{block}

\begin{block}{2. Falhas de comunicação pública}
\begin{itemize}
    \item Confusão entre informação e tributação
    \item Dados técnicos mal contextualizados
    \item Ausência de estratégia de comunicação
\end{itemize}
\end{block}

\begin{block}{3. Educação tributária insuficiente}
\begin{itemize}
    \item Carência de conhecimento sobre obrigações acessórias
    \item Dificuldade em interpretar dados estatísticos
    \item Vulnerabilidade a desinformação
\end{itemize}
\end{block}
\end{frame}

\begin{frame}{Análise do Caso}
\begin{center}
\begin{tikzpicture}[
    node distance=1cm,
    box/.style={draw, rounded corners, align=center, text width=3cm, minimum height=1.5cm}
]

\node[box, fill=azulrf!20] (intencao) {Intenção:\\ Modernização fiscal};
\node[box, fill=azulrf!30, right=of intencao] (implement) {Implementação:\\ Novos limites};
\node[box, fill=vermelho!20, below=of implement] (reacao) {Reação:\\ Desinformação};
\node[box, fill=vermelho!30, left=of reacao] (resultado) {Resultado:\\ Revogação};

\draw[->, thick] (intencao) -- (implement);
\draw[->, thick] (implement) -- (reacao);
\draw[->, thick] (reacao) -- (resultado);
\draw[->, thick, dashed] (resultado) -- (intencao);

\node[below=1.0cm of reacao, text width=8cm, align=] {\small\textcolor{red}{Falha no ciclo: Técnica desconectada da comunicação e educação}};
\end{tikzpicture}
\end{center}
\end{frame}


\begin{frame}{Recomendações}
    \begin{enumerate}
        \item \textbf{Reintroduzir a atualização} com melhor comunicação
        \item Criar \textbf{câmara técnica permanente} RFB-BCB-Setor
        \item Implementar \textbf{programa nacional de educação tributária}
        \item Desenvolver \textbf{protocolos de transparência ativa}
        \item Estabelecer \textbf{revisão automática} por indicadores econômicos
    \end{enumerate}
    
    \vspace{0.5cm}
    \begin{block}{Visão de futuro}
    \vspace{0.09cm}
    Sistema tributário digital que equilibre eficiência fiscal, inovação financeira e proteção do cidadão
    \end{block}
\end{frame}

\begin{frame}{Conclusão Final}
    \begin{center}
    \Large
    \textcolor{azulrf}{A modernização fiscal na era digital exige:}
    
    \vspace{0.5cm}
    \begin{tabular}{c}
    \hline
    \textbf{Excelência técnica} \\
    + \\
    \textbf{Comunicação eficaz} \\
    + \\
    \textbf{Educação permanente} \\
    \hline
    \end{tabular}
    
    \vspace{0.5cm}
    \textcolor{vermelho}{Falta de qualquer um = Risco de fracasso}
    \end{center}
\end{frame}


% --- Bibliografia ---
\begin{frame}{Bibliografia 1/3 - Instituições Governamentais}
    \fontsize{7}{11}\selectfont
    \bibliographystyle{abntex2-alf}
    \begin{thebibliography}{}
    
        \bibitem[BRASIL, 2025a]{receita-competencias}
        BRASIL. Receita Federal. \textbf{Competências}. Disponível em: \url{https://www.gov.br/receitafederal/pt-br/acesso-a-informacao/institucional/competencias}. Acesso em: 20 maio 2025.
        
        \bibitem[BRASIL, 2025b]{bcb-funcoes}
        BRASIL. Banco Central do Brasil. \textbf{Banco Central: o que é e o que faz?}. Disponível em: \url{https://www.bcb.gov.br/content/cidadaniafinanceira/documentos_cidadania/Cadernos_BC-Serie_Educativa_para_criancas/bancocentral.pdf}. Acesso em: 20 maio 2025.
        
        \bibitem[BRASIL, 2025c]{resolucao4656}
        BRASIL. Banco Central do Brasil. \textbf{Resolução nº 4.656, de [data]}. Disponível em: \url{https://www.bcb.gov.br/estabilidadefinanceira/exibenormativo?tipo=RESOLUÇÃO&numero=4656}. Acesso em: 20 maio 2025.
        
        \bibitem[BRASIL, 2025d]{fintechs}
        BRASIL. Banco Central do Brasil. \textbf{Fintechs}. Disponível em: \url{https://cdn-www.bcb.gov.br/estabilidadefinanceira/fintechs}. Acesso em: 20 maio 2025.
        
        \bibitem[BRASIL, 2025e]{open-finance}
        BRASIL. Banco Central do Brasil. \textbf{Open Finance}. Disponível em: \url{https://www.bcb.gov.br/en/financialstability/open_finance}. Acesso em: 20 maio 2025.
    
    \end{thebibliography}
\end{frame}

\begin{frame}{Bibliografia 2/3 - PIX e Fintechs}
    \fontsize{7}{11}\selectfont
    \bibliographystyle{abntex2-alf}
    \begin{thebibliography}{}
    
    \bibitem[BRASIL, 2025f]{fintechs-credito}
    BRASIL. Banco Central do Brasil. \textbf{Fintechs de crédito e bancos digitais}. Disponível em: \url{https://www.bcb.gov.br/conteudo/relatorioinflacao/estudosespeciais/ee089_fintechs_de_credito_e_bancos_digitais.pdf}. Acesso em: 20 maio 2025.
    
    \bibitem[BRASIL, 2025g]{pix}
    BRASIL. Banco Central do Brasil. \textbf{PIX}. Disponível em: \url{https://www.bcb.gov.br/estabilidadefinanceira/pix}. Acesso em: 20 maio 2025.
    
    \bibitem[BRASIL, 2025h]{pix-faq}
    BRASIL. Banco Central do Brasil. \textbf{PIX FAQ}. Disponível em: \url{https://www.bcb.gov.br/en/financialstability/pixfaqen}. Acesso em: 20 maio 2025.
    
    \bibitem[BRASIL, 2025i]{open-finance2}
    BRASIL. Banco Central do Brasil. \textbf{Open Finance}. Disponível em: \url{https://www.bcb.gov.br/estabilidadefinanceira/openfinance}. Acesso em: 20 maio 2025.
    
    \bibitem[BRASIL, 2025j]{e-financeira}
    BRASIL. Receita Federal. \textbf{Receita Federal esclarece evolução na e-Financeira}. Disponível em: \url{https://www.gov.br/receitafederal/pt-br/assuntos/noticias/2025/janeiro/receita-federal-esclarece-evolucao-na-e-financeira}. Acesso em: 20 maio 2025.
    
    \end{thebibliography}
\end{frame}

\begin{frame}{Bibliografia 3/3 - Legislação e Estudos}
    \fontsize{5}{6}\selectfont
    \bibliographystyle{abntex2-alf}
    \begin{thebibliography}{}
    
    \bibitem[SINESP, 2025]{hierarquia-leis}
    SINESP. \textbf{Hierarquia das Leis no Brasil}. Disponível em: \url{https://www.sinesp.org.br/images/2020/cursos-2021/anexo_Hierarquia_das_Leis_no_Brasil.pdf}. Acesso em: 20 maio 2025.
    
    \bibitem[MODELO INICIAL, 2025]{instrucoes-normativas}
    MODELO INICIAL. \textbf{Legislação brasileira: instruções normativas}. Disponível em: \url{https://modeloinicial.com.br/materia/legislacao-brasileira-instrucoes-normativas}. Acesso em: 20 maio 2025.
    
    \bibitem[CONSULCAMP, 2025]{in-rfb-2219}
    CONSULCAMP. \textbf{IN RFB nº 2.219/2024: mudanças na prestação de informações financeiras}. Disponível em: \url{https://consulcamp.com.br/2025/01/13/in-rfb-no-2219-2024-mudancas-na-prestacao-de-informacoes-financeiras/}. Acesso em: 20 maio 2025.
    
    \bibitem[BIS, 2025]{pix-banco-central}
    BANCO DE PAGAMENTOS INTERNACIONAIS (BIS). \textbf{Estudo sobre o PIX e o Banco Central}. Disponível em: \url{https://static.poder360.com.br/2022/03/bis-banco-central-pix-estudo-23mar2022.pdf}. Acesso em: 20 maio 2025.
    
    \bibitem[EMERALD, 2025]{tecnologia-pix}
    EMERALD INSIGHT. \textbf{Tecnologia e PIX}. Disponível em: \url{https://www.emerald.com/insight/content/doi/10.1108/inmr-10-2022-0133/full/html#sec002}. Acesso em: 20 maio 2025.
    
    \bibitem[BRASIL, 2025k]{estatisticas-pix}
    BRASIL. Banco Central do Brasil. \textbf{Estatísticas PIX}. Disponível em: \url{https://www.bcb.gov.br/estabilidadefinanceira/estatisticaspix}. Acesso em: 20 maio 2025.
    
    \bibitem[BRASIL, 2025l]{relatorio-pix}
    BRASIL. Banco Central do Brasil. \textbf{Relatório de Gestão PIX 2023}. Disponível em: \url{https://www.bcb.gov.br/content/estabilidadefinanceira/pix/relatorio_de_gestao_pix/relatorio_gestao_pix_2023.pdf}. Acesso em: 20 maio 2025.
    
    \bibitem[ABRAPP, 2025]{in-rfb-2219-original}
    ABRAPP. \textbf{Instrução Normativa RFB nº 2.219, de 17 de setembro de 2024}. Disponível em: \url{https://www.abrapp.org.br/legislacao/instrucao-normativa-rfb-no-2-219-de-17-de-setembro-de-2024/}. Acesso em: 20 maio 2025.
    
    \end{thebibliography}
\end{frame}

\end{document}